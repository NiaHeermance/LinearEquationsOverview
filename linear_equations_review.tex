\documentclass[12pt]{scrartcl}
% Commit: ed8a9629
\input{NiaLatexHelpers/nia_defs.tex}
\parskip=0.5cm

\title{\vspace{-2em} Linear Equations Review}
\author{}
\date{}

\begin{document}
\maketitle
\vspace{-6em}

\section*{What are linear equations?}
Linear Equations like the following
\[x + y + z = 1\]
have an $n$-dimensional plane as their graph:
\begin{figure}[H]
    \centering
    \includegraphics[scale=0.6]{x + y + z = 1.png}
    \caption*{$x + y + z = 1$}
\end{figure}
Because we have 3 variables, the graph is a $3-1=2$-dimensional plane, which is the typical plane we all know and love. We subtract 1 from the amount of dimensions because we can think of $z$ as a function of $x$ and $y,$ and that's 2 variables. So here, $z(x,y) = 1 - x - y.$ (Alternatively, we could think of $x$ as a function of $y,z$ or $y$ as a function of $x,z.$) If the graph had 2 variables initially, it would be a $2-1=1$-dimensional plane, which is a line, and if it had 4 variables, it would be a $4-1=3$-dimensional plane, which would be a cube. And if it had 5 variables, it would be a 4-dimensional plane, which whould be a hyperplane!

Just like how a line extends forever, a 2D plane as visualized here extends out in all directions. We're only looking at a small piece of it. Linear equations create $n$-dimensional planes mathematically because the first derivatives with respect to each variable is constant. But intuitively, it's because if we set any of the variables to 0, say $x$, we would get the equation $y+z = 1$, which creates a line. If $x=1$, then we would have
\begin{align*}
    1 + y + z &= 1 \\
        y + z &= 0
\end{align*}
And if $x=2,$, we would have
\begin{align*}
    2 + y + z &= 1 \\
        y + z &= -1
\end{align*}
All of these equations are lines on the original plane $x+y+z=1$ we were looking at. So you can see that a 2-dimensional plane is just a bunch of 1-d planes, aka lines, all next to each other to make a rectangle. Similarly, we could think of a 3-dimensional planes as a bunch of 2D planes stacked on top of each other, like pieces of paper stacked to make a cube. This stacking effect is one of the main reasons we call these equations linear: there's no curvature, and we can simply stack the previous dimension to get the next plane.

\begin{figure}[H]
    \centering
    \includegraphics[scale=0.4]{Planes Make the Next Dimension.png}
    \caption*{Infinite lines make a plane. Infinite planes make a cube.}
\end{figure}

Okay, that explanation is good and all, but we were just looking at $x + y + z = 1,$ where each variables has a coefficient and the final number is 1. What happens if we change the final number? Well, if $x,y=0$, we have $z=1$. Similarly, when $y,z=0,$ $x=1,$ and when $x,z=0$, $y=1.$ So the final number controls the axis intercepts.
\begin{figure}[H]
    \centering
    \includegraphics[scale=0.6]{Intercepts of Linear Equations.png}
    \caption*{$x+y+z=1\quad$ and $\quad x+y+z=3.$}
\end{figure}
Alright, so what do the coefficients do? Consider the equation $x+y+2z = 1.$ Because of the $2$ the slop is actually slower in the $z$ direction, as if we set $x=0$, we get $z = \frac{1}{2} - \frac{1}{2}y$, and if we set $y=0$, we get $z = \frac{1}{2} - \frac{1}{2}x.$ So the larger the coefficient, the smaller the relative slope in that axis direction.

\begin{figure}[H]
    \centering
    \includegraphics[scale=0.6]{Coefficient Affects Slope in that Axis.png}
    \caption*{Red is $x+y+z=1$, \: purple is $x+y+4z=1$.\quad Blue is $2z$, \: green is $3z$.}
\end{figure}

\section*{Solving a System of Equations}
Consider the following equations:
\begin{align*}
    2x + y + z &= 2 \\
    x + 2y - z &= 3 \\
    -x - 3y + 4z &= 2
\end{align*}
We know that in algebra, it's valid to add equations together and to multiply both sides of an equation by a number. The order of how we listed the equations above doesn't suddenly change the planes, so the solution will stay the same if we list them in a different order by swapping two of them. These three operations, multiplying by a number (also called a scalar), adding equations together, and swapping equations in the order we list them are valid algebraic procedures, so when we represent the system as a matrix, we are allowed to do these operations. We call them the \textit{elementary row operations}.

The matrix for the system above is:
\setlength\arraycolsep{8pt}
\def\arraystretch{1.4}
\[\bmat[ccc|c] 2 & 1 & 1 & 2 \\ 1 & 2 & -1 & 3 \\ -1 & -3 & 4 & 2 \emat\]
You may be wondering why we don't add numbers  or variables to both sides of an equation to help with solving. Our row operations only allow adding an equation to another equation, not adding numbers or variables to a single equation. However, we can of course do that! It's completely algebraically valid. It's just that, if we say, subtract 2 from equation 1,
\begin{align*}
    2x + y + z &= 2\\
    -2 \quad \: &\:\:-2 \\
    2x + y + z - 2 &= 0,
\end{align*}
well, we'll just have to add 2 again in order to get the equation in the form $ax + by + cz = d$ in order to get the top right entry in the matrix, which just means adding 2 all over again. Similarly, if we wanted to, say, solve for $z,$ we could subtract $2x$ and $y$ and get
\begin{align*}
    2x + y + z &= 2\\
    \text{$-$}2x \:\:\, \text{$-$}y\quad\:\: &\quad \: \text{$-$}2x \quad \text{$-$}y \\
   z &= 2 - 2x - y,
\end{align*}
This would be super useful if we were solving the equation with plain algebra! But because we want to solve an equation with a matrix, we would still have to convert the equation back to $ax + by + cz = d$ in order to get the entries for the first row of the matrix, undoing our progress. So it turns out that with a matrix and the elementary row operations, we can solve a system of equations without having to add numbers or variables to an equation (instead, we add whole entire equations). Adding numbers or variables to an equation is useful when you're doing algebra once you're in row-echelon form, but not before then if you want to procede with matrices, which is what we do in a linear algebra course.

So now that we know the algebraic operations useful to us (which is adding equations to another equation, multiplying by a scalar, and swapping equations in our list), we're ready to simplify the matrix! Our goal is to get to the form:
\[\begin{bmatrix}[ccc|c]
    1 & 0 & 0 & \text{some expression} \\
    0 & 1 & 0 & \text{another expression} \\
    0 & 0 & 1 & \text{a third expression}
\end{bmatrix}\]
This is because we would have $x + 0y + 0z =$ some expression, so $x =$ some expression. We would have the answer for $x$ that we could read right off the matrix! Similarly, $y =$ another expression and $z =$ a third expression. So let's begin row reducing.

When looking at the matrix, repeated for convience,
\[\begin{bmatrix}[ccc|c]
    2 & 1 & 1 & 2 \\
    1 & 2 & -1 & 3 \\
    -1 & -3 & 4 & 2
\end{bmatrix}\]
you should think about which row has its first column closest to 1. In this case, it is row 2, as its column 1 entry already is 1. So if we swap rows 1 and 2, we have the top left done.
\[
    \begin{bmatrix}[ccc|c]
        2 & 1 & 1 & 2 \\
        1 & 2 & -1 & 3 \\
        -1 & -3 & 4 & 2
    \end{bmatrix}
    \xto[R_1 \lrto R_2]
    \begin{bmatrix}[ccc|c]
        1 & 2 & -1 & 3 \\
        2 & 1 & 1 & 2 \\
        -1 & -3 & 4 & 2
    \end{bmatrix}
\]
Next, we got to make it 0 in the first column for the new row 2 and also row 3:
\begin{align*}
    \begin{bmatrix}[ccc|c]
        1 & 2 & -1 & 3 \\
        2 & 1 & 1 & 2 \\
        -1 & -3 & 4 & 2
    \end{bmatrix}&\\
    \xto[R_2 \to R_2 - 2R_1][R_3 \to R_3 + R_1]&
    \begin{bmatrix}[ccc|c]
        1 & 2 & -1 & 3 \\
        0 & 1 - 2 \cdot 2 & 1 - 2 \cdot -2 & 2 - 2 \cdot 3 \\
        0 & -1 & 3 & 5
    \end{bmatrix} \\
    =\quad\:\:\: &
    \begin{bmatrix}[ccc|c]
        1 & 2 & -1 & 3 \\
        0 & -3 & 5 & -4 \\
        0 & -1 & 3 & 5
    \end{bmatrix}
\end{align*}
Now we look at rows 2 and 3 and decide which row is closest to 1 in the second column. We see that row 3 has $-1$ in its second column, which we can easily fix by multiplying by $-1.$ So we will swap rows 2 and 3 and then multiply the new row 2 by -1. (Notice that we didn't consider swapping row 1 at all because we already have chosen row 1 to stay as the top row due to the 1 in column 1.)
\begin{align*}
    \begin{bmatrix}[ccc|c]
        1 & 2 & -1 & 3 \\
        0 & -3 & 5 & -4 \\
        0 & -1 & 3 & 5
    \end{bmatrix}
    \xto[R_2 \lrto R_3]&
    \begin{bmatrix}[ccc|c]
        1 & 2 & -1 & 3 \\
        0 & -1 & 3 & 5 \\
        0 & -3 & 5 & -4
    \end{bmatrix} \\
    \xto[R_2 \to \: -R_2]&
    \begin{bmatrix}[ccc|c]
        1 & 2 & -1 & 3 \\
        0 & 1 & -3 & -5 \\
        0 & -3 & 5 & -4
    \end{bmatrix}
\end{align*}
Now we need to get rid of the 2 in row 1 column 2 and the $-3$ in row 3 column 2. We'll get rid of the 2 later if we decide to go all the way to reduced row-echelon form. Recall that reduced row-echelon form and row-echelon form are ways of making the end algebra we have to do easier. With row-echelon form, we have to do a little easy algebra at the end, but with reduced row-echelon form, the solution is literally in the matrix so we don't need to do any algebra.
\begin{align*}
    \begin{bmatrix}[ccc|c]
        1 & 2 & -1 & 3 \\
        0 & 1 & -3 & -5 \\
        0 & -3 & 5 & -4
    \end{bmatrix}&\\
    \xto[R_3 \to R_3 + 3R_2]&
    \begin{bmatrix}[ccc|c]
        1 & 2 & -1 & 3 \\
        0 & 1 & -3 & -5 \\
        0 & 0 & 5 + 3 \cdot -3 & -4 + 3 \cdot -5
    \end{bmatrix} \\
    =\quad\:\:\:&
    \begin{bmatrix}[ccc|c]
        1 & 2 & -1 & 3 \\
        0 & 1 & -3 & -5 \\
        0 & 0 & -4 & -19
    \end{bmatrix}
\end{align*}
We've delayed creating fractions as long as possible by swapping rows. But at this point, we're forced to divide row 3 by $-4$ in order to get a $z =$ expression.



\section*{What do solutions to linear equations mean?}
\begin{align*}
    x + y + z &= 2 \\
    2x + y - z &= 3 \\
    -x + y + z &= 2
\end{align*}
We already know that 3 variable linear equations create 2D planes in 3D space. But what does the solution look like?

\end{document}